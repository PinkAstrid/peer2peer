\subsection{Bilans personnels}
\subsubsection{CHANTELOUP Marie-Astrid}
\begin{tabularx}{\textwidth}{|p{4.1cm}|X|}
    \hline
    \textbf{Points positifs} &  - Duo diversifié en termes de compétences, donnant une bonne répartition du travail\\
    \hline
    \textbf{Difficultés rencontrées} & - Difficultés de compréhension sur certains aspects de TCP\\
    \hline
    \textbf{Expérience personnelle} &  - Meilleure compréhension du fonctionnement d'un Peer to Peer \\%c'était énorme wlh\\
    \hline
    \textbf{Axes d'amélioration} & - Meilleure prise en main des concepts abordés en Réseau\\
    \hline
\end{tabularx}

\subsubsection{KANIA Milan}
\begin{tabularx}{\textwidth}{|p{4.1cm}|X|}
    \hline
    \textbf{Points positifs} &  - Approfondissement de la compréhension vis-à-vis du fonctionnement des échanges sur un réseau\\
    \hline
    \textbf{Difficultés rencontrées} & - Mise en place du select TCP\\
    \hline
    \textbf{Expérience personnelle} &  - Très enrichissant\\
    \hline
    \textbf{Axes d'amélioration} & - Amélioration de la recherche sur plusieurs mots-clés\\
    \hline
\end{tabularx}

\subsection{Bilan d'équipe}

    \begin{tabularx}{\textwidth}{|X|X|X|}
    \hline
    Points positifs & Points négatifs & Expérience\\
    \hline
    Équipe plurivalente et agréable & Contexte de projet difficile vis-à-vis du programme chronophage de 2è année & Meilleure compréhension du fonctionnement des connexions UPD et TCP et d'un système de peer-to-peer. Utilisation de sockets\\
    \hline
    \end{tabularx}