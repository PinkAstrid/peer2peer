\section{Programme Principal}
\subsection{Côté serveur}
Le serveur est lancé par le programme \verb|central_server|. Lors de son lancement,  il crée sa BDD (expliquée en partie 3). Il crée ensuite une socket DGRAM et attend d'être sollicité par un client.
\vskip 0.25cm
Lorsqu'il est sollicité, il regarde le header du message qu'il a reçu. Si celui-ci est "\verb|SEARCH|", il lance une recherche dans la BDD (voir partie 4). Si le header est "\verb|PUBLISH|", il commence l'ajout d'une nouvelle entrée dans la BDD (voir partie 5).

\subsection{Côté Client}
Le client est lancé par le programme \verb|peer|. Lors de son lancement, le processus se sépare en deux par un fork. Dans le fils, il est "actif" et dans le père, il est "passif".
\vskip 0.25cm

Dans son fonctionnement "actif", il demande à l'utilisateur ce qu'il compte faire : lancer une recherche, publier ou s'arrêter là. En fonction du choix de l'utilisateur, il lance la fonction appropriée.\\
\vskip 0.25cm

Dans son fonctionnement passif, il lance la fonction \verb|TCP_server| pour attendre une demande de connexion vers son port 2000 dans le cadre de la récupération de fichiers (voir partie 6).
