%Pour rester cohérent avec l'enseble du document, lorsque vous voulez changer de paragraphe sans créer de nouvelle subsection, utilisez la commande \vskip 0,25cm

\subsection{Fiche Projet}
\subsubsection{Contexte du projet}
Ce projet a été réalisé dans le cadre de RSA de la deuxième année du cycle ingénieur sous statut étudiant de TELECOM Nancy pour les approfondissement IL, ISS et LE. Il est à rendre pour le 9 mai 2021.

\subsubsection{Rédacteurs, commanditaires}
Ce document a été rédigé par CHANTELOUP Marie-Astrid et KANIA Milan, les deux membres de l'équipe projet. Le projet quant à lui a été commandité par CHRISMENT Isabelle, responsable de la partie réseaux du module de RSA.


\subsubsection{Présentation générale du projet}
Le but du projet est de créer une plateforme peer-to-peer. Cela consiste en deux éléments, un server et un (des) client(s) permettant trois types d'échanges:
\begin{itemize}
    \item la recherche de fichiers dans la base de données (client - serveur)
    \item la publication de fichiers dans la base de données (client - serveur)
    \item la récupération de fichiers depuis un autre client (client - client)
\end{itemize}


\subsubsection{Livrables}
Ce projet est évalué sur différents livrables :
\begin{itemize}
\item le code réalisé
\item les fonctionnalités développées
\item ce rapport de projet
\end{itemize}

\subsection{Organisation du document}

Dans la partie 2, nous présentons les fonctionnements généraux du serveur et du client.
\vskip 0.25cm
Dans la partie 3, nous présentons la structure de notre base de données (BDD).
\vskip 0.25cm
Dans la partie 4, nous expliquons nos choix de développement pour la recherche de fichier sur la base de données.
\vskip 0.25cm
Dans la partie 5, nous présentons la conception de la publications d'un fichier sur le serveur.
\vskip 0.25cm
Dans la partie 6, nous présentons la mise de la récupération d'un fichier depuis un autre client.
\vskip 0.25cm
Dans la partie 7, nous effectuons un bilan sur le projet et notre approche de celui-ci.

\subsection{Précisions légales}
Ce projet n'est pas destiné à un usage commercial, ainsi, les images présentes, notamment les images de test, ne sont pas destinées à la publication et ne sont pas toutes libre de droit.\\
\vskip 0.25cm
Cependant, le caractère strictement scolaire de ce projet nous autorise à les inclure en accord avec : 
\begin{itemize}
    \item Code civil : articles 7 à 15, article 9 : respect de la vie privée
    \item Code pénal : articles 226-1 à 226-7 : atteinte à la vie privée
    \item Code de procédure civil : articles 484 à 492-1 : procédure de référé
    \item Loi n78-17 du 6 janvier 1978 : Informatique et libertés, Article 38
\end{itemize}